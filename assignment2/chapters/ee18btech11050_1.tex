
\begin{enumerate}[label=\thesection.\arabic*.,ref=\thesection.\theenumi]
\numberwithin{equation}{enumi}
\item For a unity feedback system shown in Fig \ref{fig:ee18btech11050_fig1}
\begin{figure}[!ht]
\centering
\includegraphics[width=\columnwidth]{./figs/gvvv.eps}
\caption{}
\label{fig:ee18btech11050_fig1}
\end{figure}
having transfer function
\begin{align}
    G(s) = \frac{K}{(s+3)(s+9)(s+15)}
    \label{eq:ee18btech11050_1}
\end{align}
design the value of gain(K), for a gain margin of 50dB.

\item \solution

Gain Margin:
\begin{align}
    GM = -20\log{|G(j\omega_{pc})|}
    \label{eq:ee18btech11050_2}
\end{align}
where, $\omega_{pc}$ is the phase cross-over frequency, at which
\begin{align}
    \angle {G(j \omega_{pc})} = -180 \degree
    \label{eq:ee18btech11050_3}
\end{align}
First substitute, 
\begin{align}
    s = j\omega
\end{align}
\begin{align}
    \implies G(j\omega) = \frac{K}{(-27\omega^2+405)+j(-\omega^3+207\omega)}
    \label{eq:ee18btech11050_4}
\end{align}
Now the phase will be
\begin{align}
    \angle{G(j\omega)} = -\tan^{-1}(\frac{-\omega^3+207\omega}{-27\omega^2+405})
    \label{eq:ee18btech11050_5}
\end{align}
Solving for $\angle{G(j\omega)}=-180\degree$ gives
\begin{align}
  \omega_{pc} = 14.3875
\end{align}
Magnitude :
\begin{align}
    |G(j\omega)| = \frac{K}{\sqrt{(\omega^2+9)}\sqrt{(\omega^2+81)}\sqrt{(\omega^2+225)}}
    \label{eq:ee18btech11050_6}
\end{align}
Substituting value of $\omega_{pc}$ in \eqref{eq:ee18btech11050_2} gives
\begin{align}
    K = 16.406
\end{align}
This can be verified from fig \ref{fig:ee18btech11050_fig2}
\begin{figure}[!ht]
\centering
\includegraphics[width=\columnwidth]{./figs/part1.eps}
\caption{}
\label{fig:ee18btech11050_fig2}
\end{figure}
The following code generates Fig. \ref{fig:ee18btech11050_fig2}
\begin{lstlisting}
codes/ee18btech11050_1.py
\end{lstlisting}
\item Design the value gain (K) for a phase margin of 40\degree.
\item \solution

Phase Margin:
\begin{align}
    PM = 180\degree + \phi_{gc}
    \label{eq:ee18btech11050_7}
\end{align}
where $\phi_{gc}$ is the phase angle at the gain cross over frequency $\omega_{gc}$.
At gain cross over frequency,
\begin{align}
    |G(j\omega_{gc})| = 0
    %\label{eq:ee18btech11050_2}
\end{align}
\begin{align}
    \implies -20\log{|G(j\omega_{gc})|} = 0
    \label{eq:ee18btech11050_111}
\end{align}

Given,
\begin{align}
    PM = 40\degree = 180\degree + \phi_{gc}
    \label{eq:ee18btech11050_8}
\end{align}
\begin{align}
    \implies \phi_{gc} = -140\degree = \angle{G(j\omega_{gc})}
\end{align}
From \eqref{eq:ee18btech11050_5} 
\begin{align}
    \angle{G(j\omega_{gc})} = -\tan^{-1}(\frac{-\omega_{gc}^3+207\omega_{gc}}{-27\omega_{gc}^2+405})
    \label{eq:ee18btech11050_9}
\end{align}
\begin{align}
    \implies \omega_{gc} = 8.09623
    \label{eq:ee18btech11050_10}
\end{align}

Substituting this value in \eqref{eq:ee18btech11050_111}, we get
\begin{align}
    20\log{K} = 65.016
\end{align}
\begin{align}
    \implies K = 1781.56
    \label{eq:ee18btech11050_12}
\end{align}

This again can be verified from fig \ref{fig:ee18btech11050_fig3}
The following code generates Fig. \ref{fig:ee18btech11050_fig3}
\begin{lstlisting}
codes/ee18btech11050_2.py
\end{lstlisting}
\begin{figure}[!ht]
\centering
\includegraphics[width=\columnwidth]{./figs/part2.eps}
\caption{}
\label{fig:ee18btech11050_fig3}
\end{figure}

\item Design the value gain (K) to yield maximum peak overshoot of 20\% for a step input.

\item \solution
For given system, closed loop transfer function:
\begin{align}
    T(s) = \frac{G(s)}{1+G(s)H(s)}
\end{align}
where H(s) = 1
\begin{align}
    \implies T(s) = \frac{K}{(s+3)(s+6)(s+15)+K}
    \label{eq:ee18btech11050_7}
\end{align}
Output will be:
\begin{align}
    \implies Y(s) = \frac{1}{s}\frac{K}{(s+3)(s+6)(s+15)+K}
    \label{eq:ee18btech11050_7}
\end{align}

Maximum peak overshoot :
\begin{align}
    M_p = \frac{y(t_p) - y(\infty)}{y(\infty)}
    \label{eq:ee18btech11050_15}
\end{align}
which is given as 20\%.
Here, $t_p$ is the peak time. Solving this, we get
\begin{align}
    \implies \frac{y(t_p)}{y(\infty} = 1.2
    \label{eq:ee18btech11050_15}
\end{align}
Plotting y(t) for different values of K, we choose the value of K, which gives the above ratio, which is verfied from fig \ref{fig:ee18btech11050_fig4}.
Thus, we get 
\begin{align}
    t_p = 0.505
    \label{eq:ee18btech11050_16}
\end{align}
\begin{align}
    \implies K = 928.035
    \label{eq:ee18btech11050_17}
\end{align}
\begin{figure}[!ht]
\centering
\includegraphics[width=\columnwidth]{./figs/F.eps}
\caption{}
\label{fig:ee18btech11050_fig4}
\end{figure}

The following code generates fig \ref{fig:ee18btech11050_fig4}
\begin{lstlisting}
codes/ee18btech11050_3.py
\end{lstlisting}
